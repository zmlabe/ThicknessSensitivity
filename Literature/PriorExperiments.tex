\let\latexnofiles\nofiles
\let\nofiles\relax
\documentclass[12pt,fleqn]{article}
\usepackage[margin=1in,headsep=0.4in]{geometry}
\usepackage{graphicx}
\usepackage{amsmath, amsthm, amssymb, amsfonts}
\usepackage[sort,square]{natbib}
\usepackage{setspace}
\usepackage{url}
\usepackage{xcolor}
\usepackage{capt-of}
\usepackage{caption}
\usepackage{lineno}
\usepackage{float}
\usepackage{enumitem}
\usepackage{fancyhdr}
\usepackage[colorlinks = true,
            linkcolor = blue,
            urlcolor  = blue,
            citecolor = blue,
            anchorcolor = blue]{hyperref}
\providecommand{\e}[1]{\ensuremath{\times 10^{#1}}}
\newcommand{\etal}{\textit{et al}.}
\newcommand{\ie}{\textit{i}.\textit{e}.,\ }
\newcommand{\eg}{\textit{e}.\textit{g}.,\ }
\newcommand{\Si}{5.67032\e{-8} \frac{W}{m^{2}\cdot K^{4}}}
\newcommand{\W}{\frac{W}{m^{2}\cdot K^{4}}}
\usepackage[yyyymmdd,hhmmss, us,12hr]{datetime}
\usepackage{indentfirst}
\usepackage{listings}
\usepackage{gensymb}
\usepackage{cancel}
\usepackage{titlesec}
\titlespacing*{\section}{0pt}{0.1\baselineskip}{\baselineskip}
\usepackage{outlines}
\usepackage{enumitem}
\setenumerate[1]{label=\textbf{\Roman*.}}
\setenumerate[2]{label=\Alph*.}
\setenumerate[3]{label=\arabic*.}
\setenumerate[4]{label=\alph*.}

\pagestyle{fancy}

%%%%%%%%%%%%%%%%%%%%%%%%%%%%%%%%%%%%%%%%%%%%%%

\begin{document}
\fancyhf{}
\fancyfoot[C]{Page \textbf{[\thepage]} of 5}

\center{\Huge ``Sea Ice Thickness Perturbation Experiments''}

\begin{outline}[enumerate]
\1 Influence of sea ice on the atmosphere: A study with an Arctic atmospheric regional climate model \citep{Rinke2006}
	\2 Experiment
		\3 RCM set to different sea ice and SST lower boundary conditions
		\3 RCM = 2 RCM HIRHAM (15-year simulations, 1979-1993) experiments
			\4 Physical parameterization of ECHAM4
			\4 Lower boundary forced by SST, sea ice fraction, and thickness (updated daily)
			\4 ``Prescribed sea ice thickness influences the thermal conduction through the ice''
		\3 Seasonal means calculated from 1979-1993 with emphasis on DJF
			\4 Large latent and sensible heat fluxes for open waters
			\4 Low incoming shortwave radiative flux
	\2 Data
		\3 ECMWF reanalysis (ERA-15)
			\3 (1 - HIRHAM.era) 1981-1993 NCEP weekly SST data with GISST prior to 1981. Sea ice fraction derived from SST warmer or cooler than -1.8\degree C. Constant ice thickness of 2 m where sea ice fraction $>$ 0.5.
			\3 (2 - HIRHAM.nps) SST/sea ice data from NPS coupled ice-ocean model and driven by ERA-15 atmospheric data. Sea ice model couples with POP ocean.
	\2 Results
		\3 ``Especially inaccurate is the assumption of a fixed sea ice thickness (most Arctic RCMs use a uniform thickness of 2 m)''
		\3 NPS--ERA15
			\4 3\degree C warmer SST in Labrador/Barents Sea and less SIC. This may be a result of model smoothing at the boundary edge.
			\4 SST in Labrador Sea may have stronger influence on atmospheric circulation
			\4 NPS shows much stronger heat upward heat fluxes in Labrador and Barents Seas
			\4 NPS has reduced heat fluxes and slightly colder (1\degree C) for central Arctic
		\3 No relationship with changing storm track
		\3 ``improved lower-boundar
		y forcing contributes to only slightly improved atmospheric simulation''
		\3 Small differences in summer from changes in SIT
		\3 Changes in SIC have stronger effects than SIT
		\3 Changes in 2mT affect SLP - known as ``cold high effect''
		\3 ``So in regions of relatively thick sea ice (inner Arctic) it seems not absolutely necessary to have the exact ice thickness and its regional distribution to get a reasonable atmospheric temperature profile''
		\3 Marginal ice zone correct SIT affects entire temperature profile column
		\3 Potential influence for thicker ice in western Arctic to have broad effect on troposphere-stratosphere coupling
	\2 Issues
\1 Atmospheric response to changes in Arctic sea ice thickness \citep{Gerdes2006}
	\2 Experiment
		\3 4 experiments with varying sea ice concentration and thickness in Arctic and North Atlantic (40-years each). Climatological seasonal cycle of SST
	\2 Data
		\3 SIC/SIT from a hindcast simulation with an ocean-sea ice model for the Atlantic and Arctic Oceans forced by NCEP/NCAR (1948-1998)
		\3 Atmospheric model is GFDL AM2 (2.5\degree x 2\degree - 24 levels)
		\3 (1 - THI95) monthly sea ice conditions averaged over 1994-1996
		\3 (2 - THI65) monthly sea ice conditions averaged over 1964-1966
		\3 (3 - CON95) monthly sea ice concentration from 1994-1996 with thickness from 1948-1998 average
		\3 (4 - CON65) monthly sea ice concentration from 1964-1966 with thickness from 1948-1998 average
	\2 Results
		\3 ``Realistic sea ice thickness changes can induce atmospheric signals that are of similar magnitude as those due to changes in sea ice cover.''
		\3 Nearly 3.5\degree C change near New Siberian Islands for TAS (THI95--THI65)
		\3 Low pressure in Arctic and high pressure in North Pacific attributed to differences in SIT, but NCEP reanalysis shows the change in North Atlantic
		\3 ''Reaction of atmosphere to decreasing SIT contains a strengthening of the NAO''
	\2 Issues
		\3 Neglect influence of SST anomalies
\1 Impact of prescribed Arctic sea ice thickness in simulations of the present and future climate \citep{Krinner2009}
	\2 Experiment
		\3 AGCM for realistic SIT prescription for 1980-2007 and 2080-2090 under SRES-A1B
	\2 Data
		\3 LMDZ4 AGCM (96 x 72 x 19)
		\3 1979-2007 observed SST and SIC and GHGs
		\3 Ocean boundary from MPI-ECHAM5
		\3 Each period has one simulation with constant thickness at 3 m (C) and one simulation prescribed (V)
	\2 Results
		\3 ``Observations of heat flux through thin (*papers*) and thick sea ice (*papers*) show similar large differences''
		\3 Differences between TAS warming from SIT during present day vs. end of century
		\3 ``It is noteworthy that the strong DJF sea-level pressure change over the Eurasian Arctic coast in response to the sea ice thinning at the end of the 21\textsuperscript{st} century resembles the ``Arctic Rapid change pattern'' for the most recent years''
		\3 SIT reduction impact between V and C is greater in future during winter due to fraction of open water in this period therefore winter surface energy budget is much more strongly influenced by turbulent heat flux over leads
		\3 Weaker heat flux at end of century due to deeper snow cover
		\3 Surface temperature inversion strength is reduced by 2.5\degree C
		\3 Weak warming of mid troposphere and up
		\3 Weak circulation (500 hPa) response during present day, stronger in future and greatest in spring  (MAM) with Rossby wave train feature from Aleutian to the Azores
		\3 No storm track changes
	\2 Issues
		\3 In CMIP5, the anomaly method can cause grid points to be sea ice covered, which are ice-free in the coupled model output and contain no information about SIT or SIC.
		\3 !!!!!! Their parameterization scheme has issues - relies mostly on SIC and their parameterization of melt ponds and surface albedo. Big differences with coupled experiments.
\1 Fast atmospheric response to a sudden thinning of Arctic sea ice \citep{Semmler2016}
	\2 Experiment
		\3 Surface temperature forcing that causes rapid decrease in Arctic SIT and hence a fast atmospheric response
	\2 Data
		\3 IFS (cycle 37r3) - ECMWF model with time step of 1 hr and 91 vertical levels
		\3 15-day experiments from 1979-2012 creating ensemble of 408 pairs (forced with ERA-interim)
		\3 CTL and RED (reduced thickness)
		\3 90-day experiments to get quasi-equilibrium response from ensemble of 204 pairs
		\3 RED simulations had sea ice surface temperature (SIST) increase by 10\degree C throughout duration of simulation
			\4 If freezing point of sea ice had been exceeded, then the SIST was set to freezing point of sea water
		\3 For typical 1-2 m SIT this is a 60-70\% reduction in SIT
	\2 Results
		\3 Reduction in temperature gradient around 60\degree N is well above main baroclinic zones in storm formation regions of North Pacific and North Atlantic so therefore little effect on midlatitude storm tracks
		\3 Thinner ice accompanied by more convective heating and less large-scale precipitation
		\3 Increased SIST causes quite negative mean SLP anomaly over central Arctic, but quickly diminishes
		\3 Building positive 500 mb height anomaly over 90 days
	\2 Issues
		\3 Strong height dependence for temperature response indicates importance of models simulating the Arctic boundary layer processes
\1 Sea ice thickness and recent Arctic warming \citep{Lang2016}
	\2 Experiment
		\3 Constrain a model using SIT reanalysis 
	\2 Data
		\3 EC-Earth (based on IFS (36r4) from ECMWF with 91 vertical layers
		\3 Two sets of hindcast experiments using 10 ensembles forced by ERA-I over 1982-2013
		\3 CTRL has observed SST and SIC from OISST and GHG from CMIP5, SIT set at 1.5 m
		\3 RIT has same forcing as CTRL, but thickness distribution used from GIOMAS
	\2 Results
		\3 Up to 1.5\degree C per decade of wintertime warming from SIT
		\3 1/3 to 1/2 of wintertime warming in Eastern Arctic may be due to changes in SIT
		\3 Little warming contribution in Barents Sea
		\3 TAS warming is from thermodynamic rather than circulation-change effect
		\3 Weakening of the Iceland/Aleutian lows 
		\3 SIT contributed roughly half of SLP trend in Arctic 
	\2 Issues
		\3 Their model does not account for snow depth on sea ice
\1 Experiment from Russell Blackport (*see email*)
	\2 10-year averages of SST/SIC from a particular time period in LENS to force a AGCM
	\2 CICE model sets thickness to 2 m at each grid point in Arctic
	\2 Therefore, large difference between AGCM and LENS with cold bias in AGCM (especially along Siberia)
		\3 See the figure which is AGCM--LENS for 2 m TAS
	\2 Then computed thickness in AGCM from reading the LENS output
	\2 Reducing thickness by 0.5 m caused 2-3\degree C warming
\end{outline}

%%%%%%%%%%%%%%%%%%%%%%%%%%%%%%%%%%%%%%%%%%%%%%
\noindent\makebox[\linewidth][r]{\rule{\textwidth}{1pt}}
%%%%%%%%%%%%%%%%%%%%%%%%%%%%%%%%%%%%%%%%%%%%%%
%%%%%%%%%%%%%%%%%%%%%%%%%%%%%%%%%%%%%%%%%%%%%%%%%%%%%%%%%%%%%%%%%%%%%
% Make your BibTeX bibliography by using these commands:
\bibliographystyle{/Users/zlabe/Documents/agufull08.bst}
\bibliography{/Users/zlabe/Documents/library}

\end{document}